\documentclass[
debug-values,
cover= {
	fullcover,
	offset= true,
	inner-backtitle=false,
	outter-backtitle=true
	  },
cat = {toc-name = Patatas pata todos}, 
data= {
	alumne = {Vladimir Scvria},
	titulacio = {Software},			
	assignatura	= {Pràctiques de Procrastinació-Avançada en LaTeX},
	assig-curt = {PPAL},
	assig-codi = {01312456y},						
	activitat = {Repte 1 Introducció a xpl3},
	nomcurt = {PAC1},
	date = {today}	
	}	
 ]{../P3CTeX}%% '../' added for test
%

%\PECTeX[px-cover]{uppertitle, set-uppertitle= Patatas pata Todos, set-minititle=PAC1}

\begin{document}
%	\PECTeXsetName{toc}{Patata}
	
	\section{Introducción a las Herramientas PECTeX}
		
		La clase PECTeX proporciona una serie de comandos personalizados diseñados para automatizar y simplificar tareas comunes en la creación de documentos académicos. Esta sección detalla los comandos principales y muestra ejemplos prácticos para cada uno.
		\PXimg{pxTestIMG05}{Un grfico de ejemplo.}
				A continuación, se explican los comandos divididos en dos categorías principales:
		\begin{enumerate}
			\item Gestión de Imágenes y Capturas.
			\item Gestión de Variables Dinámicas.
		\end{enumerate}
		\subsection{Comandos para Imágenes y Capturas de Pantalla}
			
			Los comandos \texttt{\textbackslash PXshot} y \texttt{\textbackslash PXimg} permiten insertar imágenes y capturas de pantalla de forma rápida y ordenada. Además, se genera automáticamente un índice para referenciar estas imágenes en el documento.
			\begin{itemize}
				\item PXshot: Inserta capturas de pantalla desde la carpeta \texttt{screenshots}.
				\item PXimg: Inserta imágenes desde la carpeta \texttt{images}.
			\end{itemize}
			\subsubsection{Ejemplos de Uso}
			
			\textbf{Capturas de Pantalla:}
			\begin{verbatim}
				\PXshot{captura}{Una captura de pantalla.}
				\PXshot[0.5]{captura}{Una captura más pequeña.}
			\end{verbatim}
			
			\PXshot{captura}{Una captura de pantalla.} 
			\PXshot[0.5]{captura}{Una captura más pequeña.}			
			\textbf{Imágenes Genéricas:}
			\begin{verbatim}
				\PXimg{grafico}{Un gráfico de ejemplo.}
				\PXimg[0.6]{grafico}{Un gráfico más pequeño.}
			\end{verbatim}
			
			\PXimg{pxTestIMG01}{Un grafico de ejemplo.} 
			\PXimg*[0.6]{pxTestIMG02}{Un gráfico más pequeño.}
			
			\PXimg*{pxTestIMG03}{Un grafico de ejemplo.} 
			\PXimg{pxTestIMG04}{Un grafico de ejemplo.} 
			\textbf{Índices Automáticos:}
			Para generar los índices de capturas y gráficos, se utilizan:
			\begin{verbatim}
				\listofscreenshots
				\listofimages-
			\end{verbatim}
			
		\newpage
		
%		\PECTeXsetKey{alumne}{Patata}
		\listofscreenshots
		\listofimages
		 %%%%%%%%%%%%%%%%%%%%%%%%%%%%%%%%%%%%%%%%%%%%%%%%%
% Override a specific term manually


\end{document}
