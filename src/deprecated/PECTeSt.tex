\documentclass[CAT]{PECTeX}
\usepackage{PECTeXv1}


% Comandes per capçalera i metadades
%% -- IDIOMA CAT/IB/ENG Per Català, Castellà i anglès respectivament.
\idioma{CAT}    
%% -- Nom de l'estudiant
\nomEst{Vladimir Scoriae}      
%% -- Titulació
\titulac{Grau d'Enginyeria Patatera}         
%% -- Codi de l'assignatura
\codAs{01312}           
%% >>>>>>>>>>>>>>>>>>>>>>>>>>>>>>>		Nom de l'assignatura
\nomAs{Pràctiques de Patatació}        					%% 	LLARG
\nomAsAUX{PP} 						%% 	CURT
        					
%% >>>>>>>>>>>>>>>>>>>>>>>>>>>>>>>		Nom de l'activitat:
\nomAct{PVP3} 						%% 	CURT
\nomActAUX{Practica Very Patatera: Part 3} 	%% 	LLARG

%% >>>>>>>>>>>>>>>>>>>>>>>>>>>>>>>		CAMPS OPCIONALS
\PECfootAUX{PEC.TeX\\Developing}		%% Camp al peu de pagina esquerre
\PECtitleAUX{This is a testing document for PEC.TeX implementation}		%% Camp auxiliar a la portada
\subtitol{Patatas Fritas Pata Todos}			%% Subtítol per a la portada
\subsubtitol{La patata definitiva en el campo de los patatales de von kartoffen analizados en el Potato Institute de Papachussets}		%% Segon subtítol per a la portada
%% >>>>>>>>>>>>>>>>>>>>>>>>>>>>>>>	%%	DATA del document
\dataActual{\today}	%% Per defecte es la data de compilació

\begin{document}
	\PECTeXStart*
	\section{Introducción a las Herramientas PECTeX}
		
		La clase PECTeX proporciona una serie de comandos personalizados diseñados para automatizar y simplificar tareas comunes en la creación de documentos académicos. Esta sección detalla los comandos principales y muestra ejemplos prácticos para cada uno.
						\PXimg{captura}{Un gráfico de ejemplo.}
		A continuación, se explican los comandos divididos en dos categorías principales:
		\begin{enumerate}
			\item Gestión de Imágenes y Capturas.
			\item Gestión de Variables Dinámicas.
		\end{enumerate}
		\subsection{Comandos para Imágenes y Capturas de Pantalla}
			
			Los comandos \texttt{\textbackslash PXshot} y \texttt{\textbackslash PXimg} permiten insertar imágenes y capturas de pantalla de forma rápida y ordenada. Además, se genera automáticamente un índice para referenciar estas imágenes en el documento.
			\begin{itemize}
				\item PXshot: Inserta capturas de pantalla desde la carpeta \texttt{screenshots}.
				\item PXimg: Inserta imágenes desde la carpeta \texttt{images}.
			\end{itemize}
			\subsubsection{Ejemplos de Uso}
			
			\textbf{Capturas de Pantalla:}
			\begin{verbatim}
				\PXshot{captura}{Una captura de pantalla.}
				\PXshot[0.5]{captura}{Una captura más pequeña.}
			\end{verbatim}
			
			\PXshot{captura}{Una captura de pantalla.} 
			\PXshot[0.5]{captura}{Una captura más pequeña.}
			
			\textbf{Imágenes Genéricas:}
			\begin{verbatim}
				\PXimg{grafico}{Un gráfico de ejemplo.}
				\PXimg[0.6]{grafico}{Un gráfico más pequeño.}
			\end{verbatim}
			
			\PXimg{captura}{Un gràfico de ejemplo.} 
			\PXimg[0.6]{captura}{Un gráfico más pequeño.} 
			
			\textbf{Índices Automáticos:}
			Para generar los índices de capturas y gráficos, se utilizan:
			\begin{verbatim}
				\listofscreenshots
				\listofimages
			\end{verbatim}
			
		\newpage
		\listofscreenshots
		\listofimages
		 %%%%%%%%%%%%%%%%%%%%%%%%%%%%%%%%%%%%%%%%%%%%%%%%%
		\section{Comandos para Listados de Código}
		Ahora mostramos cómo incluir fragmentos de código con los nuevos comandos.
		
		\subsection{Incluir un Fichero Completo}
		Para insertar un archivo de código completo (por ejemplo, \texttt{MyStack.java}), usamos:
		\begin{verbatim}
			\PXcodefile[Java]{MyStack.java}{Implementación completa de MyStack}[myStackAll]
		\end{verbatim}
		
		\PXcodefile[Java]{MyStack.java}{Implementación completa de MyStack para poder ver cuan largo es estoy escribiendo paore es ribir espero que quede bien formateado porque sino no se como arreglarlo }[myStackAll]
		
		\subsection{Incluir un Fragmento de Líneas}
		Para solo un trozo del mismo archivo, indicamos \emph{línea inicial} y \emph{línea final}:
		\begin{verbatim}
			\PXcodechunk[Java]{MyStack.java}{11}{20}{Solo el constructor y push() de MyStack}[pushCtor]
		\end{verbatim}
		
		\PXcodechunk[Java]{MyStack.java}{11}{20}{Solo el constructor y push() de MyStack}[pushCtor]
		
		\subsection{Lista de Fragmentos de Código}
		Al igual que con \texttt{screenshots} o \texttt{images}, disponemos de:
		\begin{verbatim}
			\listofcodes
		\end{verbatim}
		
		\newpage
		\listofcodes
		
		\section{Ejecuciones en la Consola}
		
		
	\PXinOut[sayPatata.sh]{echo "patatas pata todos"}{patatas pata todos}[sayPatata]
	\PXcmd{Patata}{echo "patatas pata todos"}
	\newpage
	\listofconsoleprompts
		
		
\PXprompt{ipconfig}
{Windows IP Configuration
	Ethernet adapter Ethernet:
	
	Connection-specific DNS Suffix  . :
	IPv6 Address. . . . . . . . . . . : 2a0c:5a84:e507:d100:2072:c71d:5df5:1086
	Temporary IPv6 Address. . . . . . : 2a0c:5a84:e507:d100:a882:7fc7:cac4:b5c1
	Link-local IPv6 Address . . . . . : fe80::e03c:cd73:a74f:93cf%14
	IPv4 Address. . . . . . . . . . . : 192.168.1.147
	Subnet Mask . . . . . . . . . . . : 255.255.255.0
	Default Gateway . . . . . . . . . : fe80::1%14
	192.168.1.1
	
	Ethernet adapter Ethernet 3:
	
	Connection-specific DNS Suffix  . :
	Link-local IPv6 Address . . . . . : fe80::c06f:fea6:9c6e:311b%15
	IPv4 Address. . . . . . . . . . . : 192.168.56.1
	Subnet Mask . . . . . . . . . . . : 255.255.255.0
	Default Gateway . . . . . . . . . :
	
	Wireless LAN adapter Wi-Fi:
	
	Media State . . . . . . . . . . . : Media disconnected
	Connection-specific DNS Suffix  . :
	
	Wireless LAN adapter Conexión de área local* 1:
	
	Media State . . . . . . . . . . . : Media disconnected
	Connection-specific DNS Suffix  . :
	
	Wireless LAN adapter Conexión de área local* 2:
	
	Connection-specific DNS Suffix  . :
	Link-local IPv6 Address . . . . . : fe80::1448:7b3e:e441:4a9%10
	IPv4 Address. . . . . . . . . . . : 192.168.137.1
	Subnet Mask . . . . . . . . . . . : 255.255.255.0
	Default Gateway . . . . . . . . . :
	
	Ethernet adapter Conexión de red Bluetooth:
	
	Media State . . . . . . . . . . . : Media disconnected
	Connection-specific DNS Suffix  . :
	
	Ethernet adapter vEthernet (Default Switch):
	
	Connection-specific DNS Suffix  . :
	Link-local IPv6 Address . . . . . : fe80::7f8:729d:d9b7:65a6%64
	IPv4 Address. . . . . . . . . . . : 172.23.112.1
	Wireless LAN adapter Wi-Fi:
	
	Media State . . . . . . . . . . . : Media disconnected
	Connection-specific DNS Suffix  . :
	
	Wireless LAN adapter Conexión de área local* 1:
	
	Media State . . . . . . . . . . . : Media disconnected
	Connection-specific DNS Suffix  . :
	
	Wireless LAN adapter Conexión de área local* 2:
	
	Connection-specific DNS Suffix  . :
	Link-local IPv6 Address . . . . . : fe80::1448:7b3e:e441:4a9%10
	IPv4 Address. . . . . . . . . . . : 192.168.137.1
	Subnet Mask . . . . . . . . . . . : 255.255.255.0
	Default Gateway . . . . . . . . . :
	
	Ethernet adapter Conexión de red Bluetooth:
	
	Media State . . . . . . . . . . . : Media disconnected
	Connection-specific DNS Suffix  . :
	
	Ethernet adapter vEthernet (Default Switch):
	
	Connection-specific DNS Suffix  . :
	Link-local IPv6 Address . . . . . : fe80::7f8:729d:d9b7:65a6%64
	IPv4 Address. . . . . . . . . . . : 172.23.112.1
	Wireless LAN adapter Wi-Fi:
	
	Media State . . . . . . . . . . . : Media disconnected
	Connection-specific DNS Suffix  . :
	
	Wireless LAN adapter Conexión de área local* 1:
	
	Media State . . . . . . . . . . . : Media disconnected
	Connection-specific DNS Suffix  . :
	
	Wireless LAN adapter Conexión de área local* 2:
	
	Connection-specific DNS Suffix  . :
	Link-local IPv6 Address . . . . . : fe80::1448:7b3e:e441:4a9%10
	IPv4 Address. . . . . . . . . . . : 192.168.137.1
	Subnet Mask . . . . . . . . . . . : 255.255.255.0
	Default Gateway . . . . . . . . . :
	
	Ethernet adapter Conexión de red Bluetooth:
	
	Media State . . . . . . . . . . . : Media disconnected
	Connection-specific DNS Suffix  . :
	
	Ethernet adapter vEthernet (Default Switch):
	
	Connection-specific DNS Suffix  . :
	Link-local IPv6 Address . . . . . : fe80::7f8:729d:d9b7:65a6%64
	IPv4 Address. . . . . . . . . . . : 172.23.112.1
	Subnet Mask . . . . . . . . . . . : 255.255.240.0
	Default Gateway . . . . . . . . . :
}

\end{document}
